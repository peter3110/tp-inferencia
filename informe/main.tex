
\documentclass[a4paper]{article}

\usepackage[paper=a4paper, left=1.5cm, right=1.5cm, bottom=1.5cm, top=1.5cm]{geometry}
\usepackage[utf8]{inputenc}
\usepackage[spanish]{babel}
\usepackage{ulem}
\usepackage{amssymb, amsmath, amsbsy}
\usepackage{caratula/caratula}
\def\doubleunderline#1{\underline{\underline{#1}}}

\usepackage{tikz}
\usetikzlibrary{bayesnetWLee}

\begin{document}

\titulo{Trabajo Práctico \#1 }
\fecha{26 de Mayo de 2016}
\materia{Inferencia Bayesiana}

\integrante{Pedro Rodriguez}{197/12}{pedro3110.jim@gmail.com}
\integrante{xxx}{xxx}{xxx}
% \integrante{Pepe Sánchez}{444/12}{pepe@gmail.com}
% \integrante{Roberto Carlos}{111/10}{roberto@gmail.com}

%Carátula
\maketitle
\newpage

%Indice
\tableofcontents
\newpage

% Demás secciones
%
\section{Introducción}
% \input{introduccion.tex}
En el presente Trabajo Práctico ...


\section{Modelos}
La idea de los modelos es que ... \\
Cada una de las variables representa .. \\
Utilizamos cada una de las distribuciones porque ... \\


\subsection{Modelo 1a}
\tikz{ %
		\node[obsDisc] (m) {$m_i$} ; %
        \node[detCont, right=of m] (theta) {$\theta_i$} ; %
        \node[detCont, right=of theta] (p) {$p_i$} ; %

		\node[latentCont, above left=of theta] (C) {$C$} ;
		\node[latentCont, above right=of theta] (NC) {$NC$} ;
		\node[latentCont, right=of p] (alpha) {$\alpha$} ;

     	\edge[] {theta} {m} ; 
     	\edge[] {p} {theta} ;
     	\edge[] {C} {theta} ;
     	\edge[] {NC} {theta} ;
     	\edge[] {alpha} {p} ;
        \plate {} { (m) (theta) (p) } {$i=1 \dots cantMonedas$}; %
}
\newline
Likelihood y priors: \\
\begin{itemize}
	\item $ m_i \sim Binomial(\theta_i, cantLanzamientos) $
	\item $ C \sim Beta(k_1, k_1) $, con $k_1$ una constante grande ( $\geq 100 $ )
	\item $ NC \sim Beta(k_2, k_2) $, con $k_2$ una con constante entre 0 y 1 
	\item $ \alpha \sim Uniforme(0,cantMonedas) $
	\item $ \theta_i = p_i * C + (1 - p_i) * NC $
	\item $ p_i = (\alpha < i \leq \alpha + 1) $
\end{itemize}



\subsection{Modelo 1b}
\tikz{ %
		\node[obsDisc] (m) {$m_i$} ; %
        \node[detCont, right=of m] (theta) {$\theta_i$} ; %

		\node[latentCont, above left=of theta] (C) {$C$} ;
		\node[latentCont, above right=of theta] (NC) {$NC$} ;
		\node[latentCont, right=of theta] (alpha) {$\alpha_1$} ;

     	\edge[] {theta} {m} ;
     	\edge[] {C} {theta} ;
     	\edge[] {NC} {theta} ;
     	\edge[] {alpha} {theta} ;
        \plate {} { (m) (theta) } {$i=1 \dots cantMonedas$}; %
}
\newline
Likelihood y priors: \\
\begin{itemize}
	\item $ m_i \sim Binomial(\theta_i, cantLanzamientos) $
	\item $ C \sim Beta(k_1, k_1) $, con $k_1$ una constante grande ( $\geq 100 $ )
	\item $ NC \sim Beta(k_2, k_2) $, con $k_2$ una con constante entre $0$ y $1$
	\item $ \alpha_1 \sim Categorica(\dfrac{1}{cantMonedas}, \dots , \dfrac{1}{cantMonedas}) $
	\item $ \theta_i = (i = \alpha_1) $
\end{itemize}

\subsection{Modelo 2a}
\tikz{ %
		\node[obsDisc] (m) {$m_i$} ; %
        \node[detCont, right=of m] (theta) {$\theta_i$} ; %

		\node[latentCont, above left=of theta] (C) {$C$} ;
		\node[latentCont, above right=of theta] (NC) {$NC$} ;
		\node[latentCont, right=of theta] (alpha) {$\alpha_2$} ;

     	\edge[] {theta} {m} ;
     	\edge[] {C} {theta} ;
     	\edge[] {NC} {theta} ;
     	\edge[] {alpha} {theta} ;
        \plate {} { (m) (theta) } {$i=1 \dots cantMonedas$}; %
}
\newline
Likelihood y priors: en este caso, el likelihood y los priors son los mismos que en el modelo 1a,
cambiando $\alpha_1$ 
por $\alpha_2$:
\begin{itemize}
	\item $ \alpha_2 \sim Bernoulli(0.5) $
\end{itemize}

%
%
% \section{Conclusiones}
% \input{conclusiones.tex}
%

\end{document}

